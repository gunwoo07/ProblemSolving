\documentclass{article}

\usepackage[utf8]{inputenc}
\usepackage{kotex}

\title{What's this smell?}
\author{pg13}
\date{From 10 May 2024 to }

\begin{document}

\maketitle

\section{What}
2024년 5월 9일 밤에 나와 Ice hike player 방의 야수와 빈이 이방했다. 
그리고 김치찌개도 오기로 하였으나 우리 방문을 여는 순간 하수구 냄새가 많이 난다고 오지 않았다.
다시 김치찌개가 내 방으로 오게 만들기 위해서 PSP 기법을 이용하여 이 문제를 해결하고자 한다.
내가 정의한 문제는 화장실 하수구 냄새가 방에 난다는 것이다. 
그리고 목표 상태는 화장실을 정상적으로 사용할 수 있으면서도 방에서 하수구 냄새가 나지 않게 하는 것이다.

\section{Why}
하수구 냄새가 어디서 나는지 알기 위해서 화장실에서 냄새가 날 만한 곳은 변기, 세면대, 물 빠지는 곳이 있다.
각각의 장소를 조사해본 결과, 물 빠지는 곳에서 냄새가 가장 심하게 났고 다른 곳에서는 거의 나지 않았다.
즉, 물 빠지는 곳에서 아래층에서부터 냄새가 위로 올라오는 것이다.

\section{How}
나는 인터넷 조사와 생각을 통해 다음 2가지 방법을 얻어내었다.
\subsection{구멍 막기}
이 방법은 집에서도 아래층에서 올라오는 담배 냄새를 막기 위해 사용했던 방법으로 구멍을 물을 넣은 지퍼백을 막는 것이다.
가장 쉽게 할 수 있는 방법이기 때문에 첫번째 시도 대상이다.
과정은 간단하다.
구멍을 막을 수 있는 사이즈의 지퍼백에 물을 약간 넣고 구멍에 올려놓으면 끝이다.
그리고 씻을 때만 잠시 지퍼백을 옆으로 치우고 다 씻으면 다시 구멍을 막으면 된다.
\subsection{베이킹소다}
화학적인 방법으로 바로 구할 수 없는 준비물이 포함되어 있기 때문에 두번째 시도 대상으로 선정되었다.
인터넷을 참고해서 찾은 방법으로 하수구에 베이킹 소다 종이컵 한 컵을 부은 다음에 30분 후에 뜨거운 물을 부어주면 된다.
이 방법은 악취뿐 아니라 하수구 뚫기도 가능하다.(뭐 굳이 필요는 없지만...)

\section{Plan}

\section{Implement}

\section{Evaluate}

\end{document}